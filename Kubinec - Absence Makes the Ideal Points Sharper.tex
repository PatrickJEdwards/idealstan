\input{|"curl -L 'https://virginia.box.com/shared/static/9120uqo655t0kwesti7t3exa7zsgbwdq.tex'"}
%\include{article_header}
\title{Absence Makes the Ideal Points Sharper: Full-data IRT Models for Legislatures}
\linespread{1.5}
\begin{document}
	
	\maketitle
	
	\begin{abstract}
		I put forward a Bayesian IRT model that can handle legislators absences as a separate category of data for determining legislator ideal points. The estimation uses the concept of a hurdle model to deflate the probabilities of legislator’s votes by the probability of absences. The model produces a single set of ideal points, but utilizes different parameters for bill absence points. Compared to existing approaches, this model tends to produce more moderate estimates of US Congresspeople’s ideal points because it can model roll-call votes where there are very few opposing votes. For parliamentary data, the model provides much more precise estimates, especially in legislatures with very high rates of absence. Additionally, the model can incorporate absentions as a middle category between yes and no votes for legislatures with high rates of abstentions.
	\end{abstract}
	
	Ideal point modeling of legislatures, and increasingly diverse kinds of social actors, has become an increasingly important part of empirical work in political science. However, most models of ideal points are based on binary outcomes that reflect yes and no votes (or positions) while all other actions are recorded as missing data. The argument for doing so is that the yes and no votes contain most of the information about legislator ideal points, and that incorporating other categories such as abstentions or absences would greatly complicate the estimation without adding much benefit. In relation to the first point, I present a model that uses Bayesian Markov Chain Monte Carlo (MCMC) estimation to handle the contingency inherent in the data. In relation to the second, I show in this paper that absences and abstentions do contain important information about legislator ideal points, a contention that has become more prominent in recent research.
	
	However, I depart from the discussion of ideal point models in this recent literature by treating absences and abstentions as additional data rather than as missing yes/no votes. This distinction, while somewhat technical, has important ramifications for modeling choices. If absences and abstentions represent missing yes/no votes, then the solution is to impute how a legislator would have voted if they had not either been absent or voted to abstain. Accomplishing this result is no easy feat because multiple imputation methods require that at some level the unobserved outcomes be random, or ignorable, relative to the observed outcomes (i.e., yes/no votes). Given that legislator behavior is usually strategic and not random, I argue that the requirements of multiple imputation are unlikely to be met without relying on additional parametric assumptions. 
	
	In this situation, the difficulty of multiple imputation is not a difficulty because the observed outcomes--absence and abstention--offer potential insight into ideal points. To this end, I have decided to incorporate absences and abstentions by adding them as outcomes within a standard item-response theory (IRT) ideal point model. While the ensuing model is more complicated than a traditional ideal point model, the model estimates a single set of ideal points, which means that the ideal points will be more precisely estimated than with a model that uses only yes or no votes. By treating absences and absentions as additional observations of legislator behavior, rather than as a statistical problem to be overcome, ideal point models can produce additional insight about legislators without requiring additional data collection.
	
	In this paper I first describe the current state of ideal point modeling with an attention to the growing awareness of the problem of absences and abstentions. I then present the model formally and offer simulation results to verify the model's performance. Then I examine two different empirical applications of the model, one drawn from data from the United States Congress and the second from the parliament in Tunisia's transitional democracy.
	
	\section*{Missing Data in Ideal Point Modeling}
	
	Following the pioneering work of \textcite{poole1997} and \textcite{jackman2004}, ideal point models have become a standard feature of the analysis of legislators and increasingly other political actors. The canonical ideal point model expresses legislator preferences as distances from a latent position in an $n$-dimensional policy space \parencite{enelow198} where the distances can be calculated either as Euclidean distances (IRT) or using the Normal distribution (i.e., the NOMINATE models), although both of these approaches tend to offer very similar estimates  \parencite{carroll2009}. A fully Bayesian analysis such as \textcite{jackman2004} is able to analyze virtually all bills, but whether done in a frequentist or Bayesian framework, the focus has remained on ideal points as representations of yes/no votes.
	
	Recently there has been criticism of these approaches because the ideal points that are estimated may not be an accurate measure of a legislator's ``true" ideological stance \parencite{krehbiel2014,Caughey2016,brauninger2016}. Statistically speaking, the ideal points are the dimension of variance that best explains the observed outcomes, so there is no easy way to know whether the ideal points refer to a legislator's ideological inclinations or more to his or her political strategies. For the model I present, however, this latter interpretation is more helpful because it makes it clear why data on absences and abstentions would be important to include. To the extent that voting can be represented in a unidimensional space, this space should be able to account for the full range of legislative behavior. The ideal point may properly refer to a legislator's firm political convictions or it may reflect their desire to appear more moderate/liberal/conservative for tactical reasons. Regardless, these ideal points are by definition the lowest-error explanation of observed behavior, and for that reason they are worth studying, even if as latent variables we cannot fully explain exactly what they represent.
	
	Building on this work, political scientists have begun to look at alternative sources of data about ideal points beyond the collection of yes and no votes. A growing literature uses data generated outside of the legislature, whether via Twitter \parencite{barbera2015}, through political donations \parencite{bonica2014}, or through the legislator's prior history as a state representative \parencite{shor2011}. Most recently, \textcite{brauninger2016} and \textcite{rosas2015} have proposed methods for including abstentions and absences in parliamentary rollcall voting, while \textcite{powell2016} has presented new data on Congressional absences that shows how different types of absences may provide varying signals of legislator partisanship.
	
	None of these papers, however, attempts to directly incorporate absences or abstentions into the afore-mentioned ideal point models, which usually code absences and abstentions as missing data. \textcite{rubin2002} argue that for missing data to be ignorable, the probability that an observation is missing must not depend on the value of the observation. \textcite{rosas2015} is the first to apply this theoretical approach to rollcall voting data via an imputation model of legislative behavior framed in terms of disagreement between a party member and the party's official position.  As \textcite{rosas2015} point out, in the case of rollcall vote data, the assumption that absences and abstentions can be thought of as a plausibly random distribution yes/no votes is unlikely to be true. In \citeauthor{rosas2015}'s model, yes and no votes are imputed based on whether a party member is in agreement with his or her party given the assumption that, conditional on knowing whether a party member is in disagreement, the ensuing decision to vote yes or no only depends on observed ideal points \parencite{rubin2002}, and hence is imputable. This model provides a compelling account of how a legislator may decide to abstain in a vote; however, the model's applicability is also limited to this situation. In addition, \citeauthor{rosas2015} collapse both absences and abstentions into a single category because in their framework these actions are equal signals of party disagreement.
	
	In the broader IRT field there are extant approaches to handling missing data. \textcite{mislevy2016} provides a helpful summary of these approaches by showing how ignorability of missing data, and the crucial MAR assumption, depends on the relationship between the person ``ability" parameters (i.e., legislative ideology) and the item parameters (i.e., bill yes/no points). However, he concludes that if the missing data is due to ``intentional ommission" or to ``examinee choice", then no extant IRT missing-data model is able to meet the MAR assumption (p. 192). As I and others have argued, virtually all recorded legislative actions are intentional at some level. To correctly impute votes, we would need to imagine a counter-factual world in which some divine force pushes the legislator to be present and to cast either a yes or no vote. We can use parametric models like  \citeauthor{rosas2015} to provide a clear context for imagining this possibility, but otherwise a standard imputation model will provide biased estimates in this situation.
	
	By comparison, the model that I propose should apply to abstention and absence data more broadly, although it does not provide special insight into intra-party dynamics. The main difference is that I do not attempt to impute observations. By treating absences and abstentions as strategic actions, I can directly incorporate them into the model as additional evidence of legislators' ideal points. This change in focus allows for a complete-data IRT model that uses every category of vote (absent,abstain,yes,no) in legislative roll-call datasets.
	
	
	
	\section*{Absence-Inflated Hurdle Model}
	
	\section*{Simulation Results}
	
	\section*{Empirical Applications}
	
	\subsection*{US Congress}
	
	\subsection*{Tunisian Parliament}
	
	\section*{Discussion}
	
	\section*{Conclusion}
	
	\section*{Appendix}
	
	\section*{References}
	
	
	
\end{document}